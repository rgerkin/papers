\documentclass[12pt]{article}
\usepackage{hyperref}
\usepackage{natbib}
\usepackage{bibentry}

\begin{document}

Computational neuroscience produces models that can help to explain empirical data. 
Empirical and theoretical researchers can benefit from resources that facilitate discovery of these models and the data that informs both these and future models.   

First, there are databases of computational models and model parts.  
The BioModels database (\url{http://www.ebi.ac.uk/biomodels-main}) is a general repository of computational models of biological processes.  
The SenseLab database, ModelDB, (\cite{migliore_modeldb_2003},\url{http://senselab.med.yale.edu/modeldb}) compiles user-submitted computational models developed specifically for simulating the electrophysiological and neurochemical properties of single neurons and networks of neurons (e.g., multi-compartment Hodgkin-Huxley type conductance based models \cite{hodgkin_quantitative_1952}). 
The Physiome Model Repository \url{http://www.cellml.org/tools/pmr} is a software suite that facilitates storage and management of models, focusing on those described in CellML, a standard markup language for biological models. 
Similarly, OpenSourceBrain (\url{http://www.opensourcebrain.org/}) is a community platform for collaborative development of computational neuron and network models that utilizes open standards such as NeuroML \cite{gleeson_neuroml:_2010} to facilitate interoperability and visualization of neuroscience models developed by different researchers. 

Second, there are databases containing structured information specific to neurons and their properties. 
For example, information on the detailed shapes of neurons (i.e., their morphology) is being compiled by NeuroMorpho \url{http://neuromorpho.org} which contains user-submitted neuron morphological reconstructions made using the NeuroLucida format \cite{glaser_neuron_1990}. 
Measured electrophysiological properties of neuron types, and the metadata associated with these measurements, is catalogued in the NeuroElectro Project \url{http://neuroelectro.org}.  
Similarly, other SenseLab databases, including NeuronDB and CellPropDB (\cite{crasto_senselab:_2007},\url{http://senselab.med.yale.edu/NeuronDB/}), contain information on the ionic currents and neurotransmitters expressed by each neuron and how these are distributed with respect to neuronal morphology. 
Detailed information on ion channel subtypes, including voltage and temporal dynamics, genetic homology, and corresponding literature references is being compiled by Channelpedia (\cite{ranjan_channelpedia:_2011},\url{http://channelpedia.net/‎}), a subproject within the Blue Brain Project \citep{markram_blue_2006}. 

Third, there are databases focusing on the anatomical organization of the brain.  
BrainInfo (\url{http://braininfo.rprc.washington.edu}) provides general information about brain areas, including what they do, where they are located, and what they contain.  
The Allen Institute for Brain Sciences provides brain-wide gene expression atlases, where the expression of each of the genes in the mammalian genome has been systematically quantified throughout the brain for a number of animal species and across stages of neural development (\cite{lein_genome-wide_2007}, \url{http://brain-map.org}). 
Similarly, the Allen Institute also provides information on the anatomical connectivity of different brain regions.
Parallel to this effort is The Human Connectome Project (\url{http://www.humanconnectomeproject.org}), a large-scale effort to map complete structural and functional neural connections \textit{in vivo} in individual humans. 
BrainMap (\url{http://www.brainmap.org}) consists of a database and related software to search published functional and structural human neuroimaging experiments.  
CoCoMac (\url{http://cocomac.g-node.org}), in contrast, is focused on the primate brain, containing records of tracing studies in the macaque. 
The Brain Architecture Management System (BAMS, \cite{bota_brain_2005}, \url{http://brancusi1.usc.edu}) contains neural connectivity information across species that has been manually curated from the existing research literature.  
Finally, Cell Centered Database (\url{http://ccdb.ucsd.edu}) focuses on images from light and electron microscopy, ranging from whole brain areas to subcellular compartments.  

In addition to these neuroscience domain-specific databases are uber-databases that provide linking facilities for cross-resource data integration. 
For example, NeuroLex (\cite{larson_neurolex.org:_2013},\url{http://neurolex.org}), provides a platform for community annotation of neuron types on the basis of morphological, neurochemical, or electrophysiological properties. 
Given this wealth of neuroscience resources, the Neuroscience Information Framework (NIF, \cite{gardner_neuroscience_2008},\url{http://www.neuinfo.org}), provides tools for semantically searching across these diverse databases through the development and incorporation of neuroscience domain-specific ontologies \cite{bug_nifstd_2008,larson_ontologies_2009,hamilton_ontological_2012,imam_development_2012}. 
For example, in NIF, the search query ``mitral cell" returns a number of database records including relevant research literature from PubMed, computational models from ModelDB, and connectivity information from BAMS.  

The number and size of these database continues to grow with the collection and contribution of ever more models and data.  
These databases give computational neuroscientists a powerful tool for constraining data-driven models and serve as an alternative to conventional literature searches.  

\bibliographystyle{plainnat}
\nobibliography{shreejoy}

\end{document}
